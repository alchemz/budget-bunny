
\bbsubsection{Account List Display}{account}

The Account List screen displays the name and the current amount of the accounts that the user has added. It also displays a flag indicating whether or not an account has been marked as default. \newline 

It allows the user to delete an account, or set it as default by swiping the list item left. When an account item is tapped, the user is taken to a screen where the account details can be viewed and edited. Note that an account marked as default cannot be deleted, so when swiping left, the only available option is "Edit", which has the same functionality as tapping the account list item.

\bbsubsubsection{Application Screenshots}{account-screenshots}

\singlescreenshot{Account-1}
\doublescreenshot{Account-2}{Account-3}

\screentable{
	\header{Screen Component}
    	{Type}
        {Description}
    \row{1. Add Button}
        {Button}
        {When tapped, the screen transitions to the Add Account Screen (See section: \nameref{add-account}) if the number of accounts is less than 20. \doublenewline
        However, if the number of accounts has reached 20, it will display an error. (See section: \nameref{account-error-scenarios})} 
    \row{2. Screen Title}
    	{Label Title}
        {Localization Key: MENULABEL\_ACCOUNT}
    \row{3. No Accounts Guide}
        {Label}
        {This label is displayed when there are no accounts present. \doublenewline 
        
        This scenario theoretically should not happen under normal circumstances, as there is no way for the user to delete an account marked as default, and when the app is freshly installed, there will be one default account available. \doublenewline
        
        Localization Key: GUIDELABEL\_NO\_ACCOUNTS}
    \row{4. Default Icon}
    	{View Element}
        {An indicator showing whether or not an account is marked as default. There can only be one default account at any one time. \doublenewline
        
        Localization Key: LABEL\_DEFAULT}
    \row{5. Edit Button}
    	{Table Cell Button}
        {This button is displayed when the swiped cell is a default account. When tapped, the screen transitions to the Edit Account Screen (See section: \nameref{edit-account}) \doublenewline
        
        Localization Key: BUTTON\_EDIT}
    \row{6. Account Cell}
    	{Label}
        {This cell contains two UI labels, one that displays the account's name, and the other displaying its current amount. \doublenewline

        It has a chevron icon, indicating that the entire cell can be tapped. When tapped, the screen transitions to the Edit Account Screen (See section: \nameref{edit-account})}
}


\screentable{
	\header{Screen Component}
    	{Type}
        {Description}
    \row{7. Set Default Button}
    	{Table Cell Button}
        {This button is displayed when the swiped cell is not a default account. \doublenewline

        When tapped, the cell will slide back and will be marked as default. The previously default cell will slide too, indicating that it was updated and unmarked. \doublenewline
        
        Localization Key: BUTTON\_SET\_DEFAULT}
    \row{8. Delete Button}
    	{Table Cell Button}
        {When tapped, the account is deleted and the cell slides away to disappear. \doublenewline
        Localization Key: BUTTON\_DELETE}
}


\bbsubsubsection{Error Scenarios}{account-error-scenarios}

\errortable{
	\errorheader{Title}{Description}
    \errorrow{Maximum Number of Accounts Reached}
    	{This error is triggered when the Add a New Account button is pressed the number of accounts present in the core data has reached 20. \doublenewline
        
        Localization Key: ERRORLABEL\_ TOO\_MANY\_ACCOUNTS}
}
